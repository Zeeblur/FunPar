\def\mytitle{Workbook}
\def\myauthor{Zoe Wall}
\def\contact{40182161@napier.ac.uk}
\def\mymodule{Fundamentals of Parallel Systems (SET09109)}

\documentclass[10pt, a4paper]{article}
\usepackage[a4paper,outer=1.5cm,inner=1.5cm,top=1.75cm,bottom=1.5cm]{geometry}
\onecolumn
\usepackage{graphicx}
\graphicspath{{./images/}}
%colour our links, remove weird boxes
\usepackage[colorlinks,linkcolor={black},citecolor={blue!80!black},urlcolor={blue!80!black}]{hyperref}
%Stop indentation on new paragraphs
\usepackage[parfill]{parskip}
%% all this is for Arial
\usepackage[english]{babel}
\usepackage[T1]{fontenc}
\usepackage{uarial}
\renewcommand{\familydefault}{\sfdefault}
%Napier logo top right
\usepackage{watermark}
%Lorem Ipusm dolor please don't leave any in you final repot ;)
\usepackage{lipsum}
\usepackage{xcolor}
\usepackage{listings}
%give us the Capital H that we all know and love
\usepackage{float}
%tone down the linespacing after section titles
\usepackage{titlesec}
%Cool maths printing
\usepackage{amsmath}
%PseudoCode
\usepackage{algorithm2e}

\titlespacing{\subsection}{0pt}{\parskip}{-3pt}
\titlespacing{\subsubsection}{0pt}{\parskip}{-\parskip}
\titlespacing{\paragraph}{0pt}{\parskip}{\parskip}
\newcommand{\figuremacro}[5]{
    \begin{figure}[#1]
        \centering
        \includegraphics[width=#5\textwidth]{#2}
        \caption[#3]{\textbf{#3}#4}
        \label{fig:#2}
    \end{figure}
}

\newcommand{\figuremacroF}[5]{
	\begin{figure}[#1]
		\centering
		\includegraphics[width=#5\textwidth]{#2}
		\caption[#3]{\textbf{#3}#4}
		\label{fig:#2}
	\end{figure}
}

\lstset{
	escapeinside={/*@}{@*/}, language=C++,
	basicstyle=\fontsize{8.5}{12}\selectfont,
	numbers=left,numbersep=2pt,xleftmargin=2pt,frame=tb,
    columns=fullflexible,showstringspaces=false,tabsize=4,
    keepspaces=true,showtabs=false,showspaces=false,
    backgroundcolor=\color{white}, morekeywords={inline,public,
    class,private,protected,struct},captionpos=t,lineskip=-0.4em,
	aboveskip=10pt, extendedchars=true, breaklines=true,
	prebreak = \raisebox{0ex}[0ex][0ex]{\ensuremath{\hookleftarrow}},
	keywordstyle=\color[rgb]{0,0,1},
	commentstyle=\color[rgb]{0.133,0.545,0.133},
	stringstyle=\color[rgb]{0.627,0.126,0.941}
}

\thiswatermark{\centering \put(336.5,-38.0){\includegraphics[scale=0.8]{logo}} }
\title{\mytitle}
\author{\myauthor\hspace{1em}\\\contact\\Edinburgh Napier University\hspace{0.5em}-\hspace{0.5em}\mymodule}
\date{}
\hypersetup{pdfauthor=\myauthor,pdftitle=\mytitle}
\sloppy
\begin{document}
	\maketitle	

	\title{\textbf{Exercise 2-1}}
	\begin{lstlisting}[caption = "Multiplier.groovy"]

	void run()
	{
		def i = inChannel.read()
		while (i > 0) {
			// write i * factor to outChannel
			outChannel.write(i*factor)
			// read in the next value of i
			i = inChannel.read()
		}
		outChannel.write(i)
	}
		
	\end{lstlisting}
	
	\begin{lstlisting}[caption = "Consumer.groovy"]
	while ( i > 0 )
	{
	    //insert a modified println statement
	    println "The output is : ${i}"
	    i = inChannel.read()
	}
	\end{lstlisting}
	
	\begin{lstlisting}[caption = "RunMultiplier.groovy"]
	def processList = [ new Producer ( outChannel: connect1.out() ),
	
		//insert here an instance of multiplier with a multiplication factor of 4
			new Multiplier ( inChannel: connect1.in(),
							outChannel: connect2.out(),
							factor: 4),
			new Consumer ( inChannel: connect2.in() )
	]

	
	\end{lstlisting}
	
	\figuremacro{H}{2-1}{Output}{ - Output from Run Multiplier program.}{0.4}
	
	\title{\textbf{Exercise 2-2}}
	
	\begin{lstlisting}[caption = "ListToStream.groovy"]
	while (inList[0] != -1)
	{
		// hint: output	list elements as single integers
		for ( i in 0 ..< inList.size)outChannel.write(inList[i])
		inList = inChannel.read()
	}
	\end{lstlisting}
   
   \bibliographystyle{ieeetr}
   \bibliography{references}

\clearpage
\section{Appendix}

		
\end{document}
